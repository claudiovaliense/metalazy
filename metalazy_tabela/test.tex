\documentclass{article}
\usepackage[utf8]{inputenc}  
\usepackage{amsmath}
\usepackage{booktabs}
\usepackage{multirow}
\usepackage{verbatim}
\usepackage[ruled, vlined, linesnumbered]{algorithm2e}
\usepackage[hidelinks]{hyperref}



\begin{document}
	\section{Algorithm}
	
	\begin{algorithm}[ht]	
		\caption{SVM Baseline}
		\label{alg:Baseline}
		
		\KwIn{Files (Files train and test), tp (Tunned params), cv (k-Folds Cross Validation), metric (Metric), df (Default paramns LBD)}			
		%Y\_global = list() \;
		\ForAll{fold $\in$ Files}{
			x\_train, y\_train, x\_test, y\_test = fold\;
			model = GridSearch(svn.SVC(df), tp, cv, metric)\;
			model.fit(x\_train, y\_train)\;
			y\_pred = model.predict(x\_test)\;
			%Y\_global[index]=y\_pred\;
			f1\_macro.add(f1\_score(y\_test, y\_pred, ``f1\_macro''))\;
			f1\_micro.add(f1\_score(y\_test, y\_pred, ``f1\_micro''))\;		
		}							
		\Return mean(f1\_macro), sd(f1\_macro), mean(f1\_micro), sd(f1\_micro);
	\end{algorithm}

	O algoritmo SVM tem complexidade O(n\_samples$^3$ x n\_features)\footnote{\url{https://scikit-learn.org/stable/modules/svm.html\#complexity}}. Utilizando o \textit{TFIDF} do dataset foi possível obter a informação de samples (linhas) e features (colunas), a Tabela \ref{tab:dataset} apresenta o impacto da entrada no algoritmo SVM. Essas informações foram obtidas do fold 0, neste caso será necessário calcular o valor para os demais folds (4).
	
	\begin{table}[ht]
		\small
		\centering	
		%\vspace{0.5cm}	
		\begin{tabular}{l c c c}	
			\toprule
			\textbf{Dataset} & \textbf{Samples} & \textbf{Features} & \textbf{SVM} \\
			\midrule    				
			Stanford Tweets & 286 & 1333 & 31183743448 \\ 
			20NG & 15071 & 98230 & 3,362562409 x $10^{17}$\\
			ACM & 19914 & 48919 & 3,86799071 x $10^{17}$\\
			\bottomrule 
		\end{tabular}
		\caption{Results for dataset \textbf{Stanford Tweets}.}
		\label{tab:dataset}
	\end{table}	
	
	
	
	\section{Experiments}
	
	\begin{table}[ht]
		\small
		\centering	
		%\vspace{0.5cm}	
		\begin{tabular}{l c c c}	
			\toprule
			\textbf{Methods} & \textbf{Macro F1} & \textbf{Micro F1} & \textbf{Time (s)} \\
			\midrule    
			Metalazy artigo & 83.81 (7.52) & 83.86 (7.51) & \\
			SVM Kernel linear & 79.01 (5.14) & 79.09 (5.12) & \textbf{1.17}\\
			SVM Kernel rbf &  79.84 (5.37) & 79.94 (5.38) & 1.31\\
			Metalazy com SVM linear & 80.71 (0.55)  & 80.49 (0.26) & 56.65 \\	
			Metalazy com SVM rbf & 80.71 (0.55) & 80.77 (0.51) & 57.41\\	
			\bottomrule 
		\end{tabular}
		\caption{Results for dataset \textbf{Stanford Tweets}.}
		\label{tab:dataset_Stanford}
	\end{table}

		
	\begin{table}[t]
		\small
		\centering	
		%\vspace{0.5cm}	
		\begin{tabular}{l c c c}	
			\toprule
			\textbf{Methods} & \textbf{Macro F1} & \textbf{Micro F1} & \textbf{Time (s)} \\
			\midrule    
			Metalazy artigo &  () &  () & \\
			SVM Kernel linear &  () &  () & \\
			SVM Kernel rbf &   () &  () & \\
			Metalazy com SVM linear &  ()  &  () &  \\	
			Metalazy com SVM rbf &  () & () & \\	
			\bottomrule 
		\end{tabular}
		\caption{Results for dataset \textbf{Reut}.}
		\label{tab:dataset_reut}
	\end{table}
	




\begin{comment}
\begin{table}[t]
	\small
	\centering	
	%\vspace{0.5cm}	
	\begin{tabular}{c c c}	
		\toprule
		\textbf{Methods} & & \textbf{STANFORD\_TWEETS}\\
		\midrule    
		\multirow{2}{*}{METALAZY}& macro F1  \\ &micro F1 \\	
		\bottomrule 
	\end{tabular}
	\caption{The cardinality of $P$ and $F$ for different cities.}
	\label{tab:dataset}
\end{table}
\end{comment}



\begin{comment}
\begin{table}[t]
	\small
	\centering	
	%\vspace{0.5cm}	
	\begin{tabular}{l c c}	
		\toprule
		\textbf{} & pred:Positive & pred:Negative \\
		\midrule    
		Positive & 29 & 6\\
		Negative & 4 & 32 \\
		\bottomrule 
	\end{tabular}
	\caption{Confusion matrix svm kernel rbf.}
	\label{tab:dataset_Stanford}
\end{table}
\end{comment}

%\newpage
%\bibliography{lib}
%\bibliographystyle{ieeetr}

\end{document}